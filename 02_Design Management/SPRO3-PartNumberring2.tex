\documentclass{article}

\title{Part Numbering System 2}
\author{Felix}
\date{State: 11.10.2023}

\begin{document}
    \maketitle
    \textbf{Updated version of PartNumbering 1}
    \section*{The System - Suggestion}
    Part numbers begin with FL - indicating that they belong to our project (FL = Forklift). 
    The forklift is split into sections. For example, base, fork and top. These are just
    examples and we can adapt them when the time comes and we start designing.
    So far, the number consists out of the following: FL-A/P-XxX, where A/P is either A for assembly or P for part. 
    XxX is being replaced by the starting
    letters  of the subsection for assemblies and starting letters of Component for a part - for instance, Ba for base. (Part)
    or DP for Development Platform (Assemblie).
    
    Drawings will be exported under the same name and stored in the same folder as the part (for now)
    Giving as a formula:
    FL-A/P-XxX. This is then followed by the version number -xx. \textbf{The numbering starts with 00}. Leading to the final 
    and \textbf{general formula}:

    \begin{itemize}
        \item \textbf{FL-A/P-XxX-xx}
    \end{itemize}

    \section{Examples}
    Part number: 
    \begin{itemize}
        \item FL-P-Ba-00 - Project forklift - Part - Base - version 0 (meaning original)
        \item FL-A-MF-04 - Project forklift - Assemblie - Mast Fork - version 4
    \end{itemize}

    \section*{Further Suggestions}
    If you have suggestions just let me know!    

\end{document}